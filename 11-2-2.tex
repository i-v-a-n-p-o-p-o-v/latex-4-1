\documentclass[8pt]{article}

\usepackage[utf8]{inputenc}
\usepackage[a5paper, left=0.2cm, bottom=0cm, right=0.2]{geometry}
%% Russian language support
\usepackage[T2A]{fontenc}
\usepackage[utf8]{inputenc}
\usepackage[russian]{babel}

\usepackage{multirow}

%% Figures
\usepackage{tkz-euclide}
\usepackage{subcaption}
\usepackage{amsmath}

%% Hyphenation rules
\usepackage{hyphenat}
\hyphenation{ма-те-ма-ти-ка вос-ста-нав-ли-вать}

\usepackage{pdflscape}

\begin{document}
% \thispagestyle{empty}

\begin{landscape}
\centering
\begin{tabular}{ p{3cm} | p{5cm}|p{5cm}|p{5cm}}
\hline
Период & Концепция использования информации&Вид &Цель использования \\
\hline
1950 - 1960 гг.& Бумажный поток & ИС обработки расчетных документов на электромеханических бухгалтерских машинах & Повышение скорости. Упрощение проедуры обработки отчетов\\
1960 - 1970 гг.& Подготовка отчетов& Управленчиские ИС для производственной информации & Ускорение процесса отчетности\\
1970 - 1980 гг.& Управленчиский контроль реализации& Системмы поддержки принятия решений. Системы для высшего звена управления & Выработка наиболее рационального решения\\
с 1980 гг.& Информация - стратегический ресурс конкуренции & Страгетические ИС. Автоматизированные офисы & Выживание и процветание фирмы\\
\end{tabular}
\end{landscape}
\end{document}