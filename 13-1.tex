\documentclass[12pt]{article}
\usepackage[T2A]{fontenc}
\usepackage[utf8]{inputenc}
\usepackage[english,russian]{babel}

\usepackage{amsmath,amssymb,amsthm,amsfonts,amscd}
\usepackage[top=2.5cm, bottom=2cm, left=3cm, right=1.5cm]{geometry}
\usepackage{indentfirst}

\linespread{1.35}
\parindent = 1cm


\begin{document}
    
    \section{Работа с индексами}
    Формула серной кислоты: $H_2SO_4$

    Квадратичная функция $y(x) = ax_2+bx+c$

    Полином степени $n \in \mathbb{N} $:

        $$a_nx^n+a_{n-1}x^{n-1}+...+a_2x^2+a_1x+a_0$$


    Очень большое число:
    $$100^{200^{300}}$$

    \section{Дроби}
    
    Аликвотная дробь:
    $$ \frac{2}{3} = \frac{1}{2} +  \frac{1}{3} =  \frac{1}{4} +  \frac{1}{4}+  \frac{1}{6}.$$

    Масштабируемое оформление дроби:
    $$  \frac{1+ \frac{x}{2}}{ \frac{y^2}{3}} $$.

    Немасштабируемо офрмление дроби:
    $$  \frac{1+ \displaystyle\frac{x}{2}}{ \displaystyle\frac{y^2}{3}} $$

    \section{Корни, функции, греческие символы}

    Для $x \geqslant 1$ и $n \in \mathbb{N}, n \geqslant 2$, справедливо:
    $$
        \sqrt{x} > \sqrt[3]{x} > \sqrt[4]{x} > ... > \sqrt[n]{x}
    $$

    Сложная функция:
    $$
        F(\omega,x) = \sqrt{\frac{|sin(\omega x)|+1}{3e^{-x}}}
    $$

    \section{Скобки}
    Вычислить:
    $$
    \left (1+\left [\frac{3x}{5}+\frac{x}{2}\right ] \right )^2,
    $$
    где $x$ - целое неотрицательное число (квадратные скобки означают взятие целой части).

    \section{Суммирование и интегрирование}

    Гёльдеровы нормы $n$-мерных векторов:
    $$
        ||x||_p = \left(  \sum_{k}|x_k|^p\right)^\frac{1}{p}
    $$
    \textit{\textbf{Теорема стокса.} Пусть на ориентируемом многообразии M размерности n заданы ориентируемое p-мерное подмногообразие $\sigma$ и дифференциальная форма $\omega$ степени p-1 класса $C^1$ $(1\leqslant p\leqslant n)$. Тогда если граница подмоногообразия $\delta\sigma$ положительно ориентирована, то
    $$
    \int_\sigma d\omega = \int_{\delta\sigma}\omega,
    $$
    где d$\omega$ обозначаетвнешний дифференциал формы $\omega$.
    }
\end{document}