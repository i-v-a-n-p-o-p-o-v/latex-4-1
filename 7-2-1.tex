\documentclass{article}

\usepackage[T2A]{fontenc}
\usepackage[utf8]{inputenc}
\usepackage[russian]{babel}

\usepackage[a4paper]{geometry}

%% Figures
\usepackage{tkz-euclide}
\usepackage{subcaption}
\usepackage{amsmath}
\usepackage{amssymb}

%% Hyphenation rules
\usepackage{hyphenat}

\usepackage[hidelinks]{hyperref}

%%colorfull
\usepackage{xcolor}

\usepackage{indentfirst}

\linespread{1.3}     % Междустрочный интервал
\parindent = 0.75cm   % Красная строка


\begin{document}
    Междустрочный интервал во всем документе равен 1.3 пункта, отступ красной строки 0.75 см.

    Это текст абзаца 2.

    Это текст абзаца 3.

    \noindent Это текст абзаца 4. Для него осуществить подавление отступа красной строки.

    \vspace{1.5cm}
    Между этим и предыдущим абзацем установить вертикальный отступ в 1.5 см.

    В этом\\ абзаце требуется\\ установить разрыв\\ через каждые\\ два слова.
    
    \centering
    Это предложение должно быть\\ отображено именно\\ так!
    
    \vspace{1.5cm}
    Для этого предложения\\ уменьшите междустрочный интервал\\ до 1.0.

    \vspace{1cm}
    Посреди этого предложения\hspace{3cm} установили большой пробел.
\end{document}